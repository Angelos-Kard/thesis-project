\pagestyle{plain}
\begin{center}
{\LARGE Extensive English Summary}\\[1cm]
\end{center}

\setlength{\parindent}{0pt}
This thesis aims to create an application for the mixed reality device Microsoft HoloLens 2. Its goal is to support individuals with visual impairments, contributing to their independence and accessibility. Specifically, it offers the ability to safely navigate an unfamiliar environment.

Initially, the theoretical and technological background is described. More specifically, we present the causes that lead to vision loss, its impact on the individual, and the available solutions to address it. Additionally, the concepts of virtual, augmented, and mixed reality are explained, and the technical characteristics of the Microsoft HoloLens 2 device and its capabilities, such as spatial mapping and spatial sound, are analyzed, along with the tools used for the application development, such as Unity.

Next, the implementation of the application is presented, starting with its design and the constraints that were set, followed by the way the project was organized and prepared. A detailed description of all the application's functions follows, with the main ones being obstacle detection and user warning, and the way they were developed.

The application evaluation was carried out with the help of ten volunteers, who were asked to test it alongside a widely used tool for visually impaired individuals, the cane. From the tests using both tools by the volunteers, we manage to collect significant comparative data, which are analyzed along with the responses from the questionnaires and the users' observations.

Finally, the thesis was completed by outlining possible future extensions and improvements of the application to provide an even better experience for the end-user.
\\[\baselineskip]
\textbf{Keywords}: {\keywordsEnglish}