%!TEX root = ../main.tex

Σκοπός του κεφαλαίου αποτελεί η παρουσίαση της διαδικασίας ανάπτυξης της εφαρμογής για τη συσκευή Microsoft HoloLens 2. Πιο συγκεκριμένα, θα παρουσιάσουμε αρχικά τον κύριο στόχο της εφαρμογής, ποιος είναι ο σκοπός που εξυπηρέτει και σε ποιο πρόβλημα προσπαθεί να προσφέρει λύση (\hyperref[sec:appScenario]{Κεφάλαιο~\ref*{sec:appScenario}}). Έπειτα, θα γίνει αναφορά στις αποφάσεις που λήφθηκαν κάτα την σχεδίαση και υλοποίηση της εφαρμογής με βάση και τους περιορισμούς που τέθηκαν (\hyperref[sec:appDesignAndLimitations]{Κεφάλαιο~\ref*{sec:appDesignAndLimitations}}). Το κεφάλαιο θα ολοκληρωθεί με μια αναλυτική περιγραφή της διαδικασίας υλοποίησης της εφαρμογής και επεξήγηση του κώδικα που αναπτύχθηκε, καθώς και με την παρουσίαση όλων των κύριων λειτουργιών (\hyperref[sec:appImplementation]{Κεφάλαιο~\ref*{sec:appImplementation}}).
% Το κεφάλαιο θα ολοκληρωθεί παρουσιάζοντας όλες τις κύριες λειτουργίες, οι οποίες αναπτύχθηκαν για την εφαρμογή (\hyperref[sec:appFunctionalities]{Κεφάλαιο~\ref*{sec:appFunctionalities}}).

%TODO: Εναλλακτικός τίτλος: Περιγραφή/Λειτουργίες Εφαρμογής
\section{Σενάριο Εφαρμογής}\label{sec:appScenario}
Τα άτομα με προβήματα όρασης έρχονται καθημερινά αντιμέτωπα με ένα πρόβλημα, το οποίο, για το μέσο πληθυσμό αποτελεί μια απλή καθημερινή ενέργεια: η απροβλημάτιστη μετακίνηση και η περιήγηση σε δημόσιους και ιδιωτικούς χώρους. Ως τώρα, παρά την τεχνολογική εξέλιξη και την ανάπτυξη σύγχρονων λύσεων, όπως αυτές περιγράφηκαν στο \hyperref[subsec:visionSolutions]{Κεφάλαιο~\ref*{subsec:visionSolutions}}, για την πλειονότητα αυτών των ατόμων, κύριος υποστηρικτής στην προσπάθεια αυτή αποτελεί το μπαστούνι, λόγω της απλότητας χρήσης του και του κόστους του.

Βασικός στόχος της εφαρμογής αποτελεί η παροχή βοήθειας σε τυφλούς και άτομα με προβλήματα όρασης, κατά την περιήγησή τους σε ένα αγνωστο προς αυτούς χώρο με ευκολία και ασφάλεια, χωρίς τη βοήθεια κάποιου επιπλέον οργάνου. Δηλαδή, η εφαρμογή πρέπει να προσφέρει ένα απλό και εύκολα κατανοητό τρόπο λειτουργίας, η οποία θα προσομοιάζει τον τρόπο χρήσης ενός μπαστονιού.

Αρχικά, ο χρήστης φορά το headset μικτής πραγματικότητας Microsoft Hololens 2 και επιλέγει την εφαρμογή που αναπτύξαμε. Κατά την εκκίνηση, καθώς και σε τακτά χρονικά διαστήματα κατά την εκτέλεση της εφαρμογής, πραγματοποιείται χωρική χαρτογράφηση του περιβάλλοντα χώρου. Με αυτό τον τρόπο δημιουργείται ένα mesh, ένα 3D μοντέλο του χώρου.

Στη συνέχεια, καθώς ο χρήστης περπατά στο χώρο, προσπαθούμε να εντοπίσουμε, σε περιοχή γύρω από αυτόν, πιθανά εμπόδια. Σε περίπτωση που η συσκευή ανιχνεύσει κάποιο εμπόδιο, προειδοποιεί τον χρήστη με τρεις διαφορετικούς τρόπους. Αναπαράγει, αρχικά, μια σύντομη ηχητική ειδοποίηση, η οποία διαθέτει ιδιότητες χωρικού ήχου. Έτσι, ο χρήστης μπορεί να αναγνωρίσει την `εικονική' θέση προέλευσης του ήχου, που αποτελεί και θέση του εμποδίου, καθώς και την απόστασή του από αυτήν. Παράλληλα, προβάλλονται οπτικά προειδοποιητικά σήματα στο πεδίο ορατότητας του ατόμου που επιδεικνύουν την θέση του εμποδίου. Τα σήματα αυτά μπορούν να αξιοποιηθούν από άτομα που αντιμετωπίζουν μερική απώλεια όρασης. Τέλος, προαιρετικά, προσφέρεται απτική ανάδραση προς το χρήστη, το οποίο του επιτρέπει να αντιληφθεί μόνο την θέση του εμποδίου στον οριζόντιο άξονα, αν θεωρήσουμε σύστημα αξόνων με αρχή τον χρήστη, δηλαδή πόσο αριστερά ή δεξιά του βρίσκεται.

Τέλος, αναπτύχθηκε ένας πρόσθετος τρόπο εντοπισμό εμποδίων. Ο εντοπισμός δε γίνεται σε συγκεκριμένη περιοχή γύρω από το χρήστη, όπως περογράφηκε προηγουμένως. Αντιθέτως, ο ίδιος προτάσσει τα χεριά του προς την κατεύθυνση που επιθυμεί να ελέγξει.

Τέλος, πρέπει να τονιστεί ότι, λόγω της ιδιαιτερότητας των χρηστών, όλες οι λειτουργίες εντός της εφαρμογής πραγματοποιούνται με χρήση φωνητικών εντολών.

\section{Σχεδιασμός και Περιορισμοί}\label{sec:appDesignAndLimitations}
Αποφάσεις για το τι ήταν ή δεν ήταν εφικτό να υλοποιηθεί

\section{Υλοποίηση}\label{sec:appImplementation}
Η υλοποίηση της εφαρμογής, τμηματοποιημένη

\subsection{Εισαγωγή}\label{subsec:developIntro}
Περιγραφή του development phase
Κώδικας των components στο παράρτημα Α μαζί με σχόλια
Όλο το project στο GitHub
Προγραμματισμός στο σπίτι και στο εργαστήριο

\subsection{Προετοιμασία και Εγκατάσταση}\label{subsec:developSetup}
Εργαλεία που χρησιμοποιήθηκαν και οι εκδόσεις τους (Unity, MRTK, Visual Studio, Hololens emulator)

\subsection{Οργάνωση GameObjects και Components}\label{subsec:developOrganization}
Ποια είναι τα GameObjects που χρησιμοποιήσαμε και ποια τα componetns που δημιουργήσαμε
Πως επιλέξαμε ποια components θα πάνε σε ποια game objects

\subsection{Εντοπισμός Εμποδίων}\label{subsec:developObstacleDetection}
Χωρική χαρτογράφηση (χαρτογράφηση γραφείου)
Raycasts και Boxcasts

\subsection{Προειδοποίηση Χρήστη}\label{subsec:developWarning}
Χωρικός ήχος
Χρήση πολλών ειδοποιήσεων - περιορισμός σε ένα
Τοποθέτηση του ήχου στο ύψος του αυτιού
Χρήση φωτός
Χρήση χειριστηρίου

\subsection{Φωνητικές Εντολές}\label{subsec:developVoiceCommands}
Χρήση φωνητικών εντολών για διάφορες λειτουργίες

% \section{Λειτουργίες Εφαρμογής}\label{sec:appFunctionalities}
% Τρόπος χρήσης της εφαρμογής

% \begin{theorem}[Shannon-Nyquist]
% 	\label{thrm:shannon-nyquist}
% 	Ένα σήμα με μέγιστη συχνότητα $f_{max}$ μπορεί να ανακτηθεί από τα δείγματά του, αν αυτά ληφθούν με συχνότητα $f_s>2f_{max}$, ή αλλιώς με περίοδο $T_s<\frac{1}{2f_{max}}$. \cite{proakis_sampling}
% \end{theorem}