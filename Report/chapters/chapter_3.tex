%!TEX root = ../main.tex

Σκοπός του κεφαλαίου αποτελεί η παρουσίαση της διαδικασίας ανάπτυξης της εφαρμογής για τη συσκευή Microsoft HoloLens 2. Πιο συγκεκριμένα, θα παρουσιάσουμε αρχικά τον κύριο στόχο της εφαρμογής, ποιος είναι ο σκοπός που εξυπηρέτει και σε ποιο πρόβλημα προσπαθεί να προσφέρει λύση (\hyperref[sec:appScenario]{Κεφάλαιο~\ref*{sec:appScenario}}). Έπειτα, θα γίνει αναφορά στις αποφάσεις που λήφθηκαν κάτα την σχεδίαση και υλοποίηση της εφαρμογής με βάση και τους περιορισμούς που τέθηκαν (\hyperref[sec:appDesignAndLimitations]{Κεφάλαιο~\ref*{sec:appDesignAndLimitations}}), ενώ θα δοθεί και μια αναλυτική περιγραφή της διαδικασίας υλοποίησης της εφαρμογής και επεξήγηση του κώδικα που αναπτύχθηκε (\hyperref[sec:appImplementation]{Κεφάλαιο~\ref*{sec:appImplementation}}). Το κεφάλαιο θα ολοκληρωθεί παρουσιάζοντας όλες τις κύριες λειτουργίες, οι οποίες αναπτύχθηκαν για την εφαρμογή (\hyperref[sec:appFunctionalities]{Κεφάλαιο~\ref*{sec:appFunctionalities}}).

%TODO: Εναλλακτικός τίτλος: Περιγραφή/Λειτουργίες Εφαρμογής
\section{Σενάριο Εφαρμογής}\label{sec:appScenario}
Στόχος εφαρμογής\\
Περιγραφή σεναρίου χρήσης εφαρμογής

\section{Σχεδιασμός και Περιορισμοί}\label{sec:appDesignAndLimitations}
Αποφάσεις για το τι ήταν ή δεν ήταν εφικτό να υλοποιηθεί

\section{Υλοποίηση}\label{sec:appImplementation}
Η υλοποίηση της εφαρμογής, τμηματοποιημένη

\section{Λειτουργίες Εφαρμογής}\label{sec:appFunctionalities}
Τρόπος χρήσης της εφαρμογής

% \begin{theorem}[Shannon-Nyquist]
% 	\label{thrm:shannon-nyquist}
% 	Ένα σήμα με μέγιστη συχνότητα $f_{max}$ μπορεί να ανακτηθεί από τα δείγματά του, αν αυτά ληφθούν με συχνότητα $f_s>2f_{max}$, ή αλλιώς με περίοδο $T_s<\frac{1}{2f_{max}}$. \cite{proakis_sampling}
% \end{theorem}