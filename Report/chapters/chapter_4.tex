%!TEX root = ../main.tex

Σκοπός της αξιολόγησης (γιατί κάνουμε αξιολόγηση)
% Συμμετέχοντες\\
% Προδιαγραφές (πως προετοιμαστήκαμε με τους εθελοντές για την αξιολόγηση)\\
% Ερωτηματολόγια που χρησιμοποιήθηκαν


\section{Διαδικασία αξιολόγησης}

\subsection{Περιγραφή Πειράματος}
Αναλυτική περιγραφή των πειραμάτων που θα διεξαχθούν\\
Ανάλυση περιορισμών > Ως προς το χώρο, τους συμμετέχοντες κλπ

\subsection{Προδιαγραφές Αξιολόγησης}
Αριθμό συμμετεχόντων / Περιγραφή αυτών\\
Μετρικές (Χρόνοι, Πλήθος συγκρούσεων), ερωτηματολόγια που θα χρησιμοποιηθούν
% Τρόπος/συνθήκες διεξαγωγής\\
% Πιθανοί περιορισμοί

\subsection{Διεξαγωγή Πειράματος}
Χώρος διεξαγωγής

\section{Αποτελέσματα}
Πίνακας με τους καταγεγραμμένους χρόνους\\
Στατιστική ανάλυση των αποτελεσμάτων και για τα δύο πειράματα\\
Παρατηρήσεις\\
Απαντήσεις ερωτηματολογίου\\
Απαντήσεις ανοιχτών ερωτήσεωνσ