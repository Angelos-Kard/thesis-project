%!TEX root = ../main.tex

Σκοπός της αξιολόγησης (γιατί κάνουμε αξιολόγηση)\\
Συυμμετέχοντες\\
Προδιαγραφές (πως προετοιμαστήκαμε με τους εθελοντές για την αξιολόγηση)\\
Ερωτηματολόγια που χρησιμοποιήθηκαν


\section{Διαδικασία/Περιγραφή Πειράματος}
Τρόπος/συνθήκες διεξαγωγής\\
Πιθανοί περιορισμοί

\section{Αποτελέσματα}
Παρατηρήσεις\\
Απαντήσεις ερωτηματολογίου

% \begin{equation} \label{eq:hrmodel}
% \bm{Y}_i=\bm{S}_i\bm{T}_i\bm{H}_i\bm{X}+\bm{n}_i
% \end{equation}

% \begin{itemize}
% 	\item $\bm{S}_{i}$ για υποδειγματοληψία, 
% 	\item $\bm{T}_{i}$ για μετατόπιση, 
% 	\item $\bm{H}_{i}$ για θόλωμα (blurring), 
% 	\item $\bm{n_{i}}$ για προσθετικό θόρυβο.
% \end{itemize}

% \begin{algorithm}
%  \KwData{$\bm{dx}$, $\bm{dy}$, $\bm{Y}_i$, $N$, $W$, $\bm{H}$, $S$}
%  \KwOut{High resolution reconstructed $\bm{X}$}
%  $\bm{X} \gets 0$\; 
%  \For{$n \gets 1$ \textbf{to} $N$}{
%  	$\bm{\tilde{Y}} \gets \bm{Y}_i$\;
%     $\bm{i} \gets 1:W/S$\;
%     $\bm{j} \gets 1:H/S$\;
%     $\bm{px}=\bm{i}*S+\bm{dx}_n$\;
%     $\bm{py}=\bm{j}*S+\bm{dy}_n$\;
%     $\bm{X}_{px,py}=\bm{\tilde{Y}}_{i,j}$\;
%     }
%     %\vspace{-1.5em}
%  \caption{Ανακατασκευή shift-add fusion}\label{algo:sr_fusion}
% \end{algorithm}
