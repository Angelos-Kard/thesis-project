%!TEX root = ../main.tex

% \markboth{Εισαγωγη}{}
%\vspace{-1.3in}
\setlength{\parindent}{24pt}
Ο άνθρωπος, από τη γέννησή του, διαθέτει πέντε βασικές αισθήσεις: την αφή, την όραση, την ακοή, την όσφρηση και την γεύση~\cite{bradford_2017_the}. Οι αισθήσεις αυτές του δίνουν τη δυνάτοτητα να αντιλήφθει καλύτερα το περιβάλλον γύρω του και συμβάλλουν σημαντικά στην επιβίωσή του. Ωστόσο, εξαιτίας ποικίλων παραγόντων, είναι πιθανό να εμφανιστεί κάποια δυσλειτουργία στο αισθητήριο όργανο, οδηγώντας σε μερική ή πλήρη απώλεια της συγκεκριμένης αίσθησης. Στα πλαίσια της συγκεκριμένης διπλωματικής εργασίας, θα εστιάσουμε στο πρόβλημα της μερικής ή πλήρης απώλειας όρασης. Η απώλεια της συγκεκριμένης αίσθησης μπορεί να δυσχεράνει την υλοποίηση καθημερινών εργασιών από το άτομο, την πρόσβαση και πλοήγηση του σε χώρους, να επηρέασει την ψυχολογία και την αυτοπεποίθησή του, οδηγώντας, τελικά, σε υποβάθμιση της ποιότητας ζωής του και έχοντας αρνητικό αντίκτυπο στην κοινωνική του ζωή και στην πνευματική του υγεία~\cite{worldhealthorganization_2023_blindness}\cite{maaikelangelaan_2007_quality}. Επόμενως, κρίνεται απαραίτητη η ανάπτυξη και η κατασκευή συσκευώνν, οι οποίες έχουν σκοπό να αναπληρώσουν την απούσα αίσθηση, εκμεταλλευόμενες τις ήδη υπάρχουσες αισθήσεις, με τελικό στόχο την διευκόλυνση της καθημερινότητας του ατόμου.

Παράλληλα, τα τελευταία χρόνια, η ανάπτυξη της τεχνολογίας αποδεικνύεται να είναι ραγδαία. Ειδικότερα, η τεχνολογία της επαυξημένης (Augmented Reality, AR), εικόνικης (Virtual Reality, VR) και μικτής (Mixed Reality, MR) πραγματικότητας <<εισβάλει>> όλο και περισσότερο στην καθημερινότητα μας, βρίσκοντας εφαρμογή σε διάφορους τομείς αυτής, όπως είναι η διασκέδαση, η εκπαίδευση, η υγεία~\cite{morgan_2021_the} κ.α.~\cite{wikipediacontributors_2019_mixed}, διαθέτοντας συσκευές και εφαρμογές προσβάσιμες στο μέσο χρήστη, λόγω της απλότητας χρήσης τους και του κόστους τους. Παραδείγματα αυτών αποτελούν οι συσκευές Meta Quest, Valve Index (γυαλιά Εικονικής Πραγματικότητας), Microsoft HoloLens και Google Glass (γυαλιά Επαυξημένης και Μικτής Πραγματικότητας). % Θα ετσιάσουμε στη συσκευή μικτής πραγματικότητας Microsoft HoloLens, η οποία αποτελεί το κύριο εργαλείο ανάπτυξης της εφαρμογής

Λαμβάνοντας υπόψην, λοιπόν, την τρέχουσα τεχνολογική πρόοδο και το πλήθος συσκευών που βρίσκονται στη διάθεση του μέσου χρήστη, καθώς και τα προβλήματα προσβάσιμότητας που αντιμετωπίζουν άτομα με μερική ή πλήρη απώλεια όρασης, τότε εύλογα τίθεται το ερώτημα:


\begin{quote}
    \textit{Είναι εφικτή η ανάπτυξη μιας εφαρμογής, η οποία  αξιοποιεί τις διαθέσιμες τεχνολογίες και hardware επαυξημένης/μικτής πραγματικότητας και παράλληλα βοηθά άτομα με αναπηρία να πραγματοποιήσουν καθημερινές εργασίες με ευκολία;}
\end{quote}
Σκοπός της διπλωματκής εργασίας είναι η υλοποιήση μιας τέτοιας εφαρμογής, η οποία θα εξυπηρετεί ειδικότερα άτομα με μερική ή πλήρη απώλεια όρασης. Η εφαρμογή στοχεύει στο να προσφέρει βοήθεια κατά την πρόσβαση και περιήγηση ενός τέτοιου ατόμου σε χώρους, ειδοποιώντας τον για πιθανά εμπόδια που μπορεί να συναντήσει στη διαδρομή του. Για τον εντοπισμό των εμποδίων και την ενημέρωση του χρήστη για αυτά, θα αναπτυχθεί λογισμικό, το οποίο θα αξιοποιεί τους αίσθητηρες και τις ενσωματωμένες τεχνολογίες της συσκευής HoloLens 2 της Microsoft.

\section{Δομή Διπλωματικής Εργασίας}
Η παρούσα διπλωματική εργασία χωρίζεται σε 5 κεφάλαια. Στο 2ο κεφάλαιο παρουσιάζεται σε βάθος το θεωρητικό και τεχνολογικό υπόβαθρο, το οποίο σχετίζεται με την ανάπτυξη της εφαρμογής. Έπειτα, στο 3ο κεφάλαιο, περιγράφεται αναλυτικά η διαδικάσια σχεδιασμού και υλοποίησης της εφαρμογής, ενώ, στο 4ο κεφάλαιο, η εφαρμογή διατίθεται σε χρήστες, με σκοπό να την αξιολογήσουν. Παρατίθονται οι συνθήκες κάτω από τις οποίες έγινε η χρήση της εφαρμογής, ο τρόπος αξιολόγησης αυτής, καθώς και τα αποτελέσματα που προέκυψαν από τα πειράματα. Τέλος, στον 5ο κεφάλαιο, αναφέρονται ορισμένες από τις δυνατότητες και προεκτάσεις, που θα μπορούσε να αποκτήσει η εφαρμογή με την περαιτέρω ανάπτυξή της και την ενσωμάτωση επιπλέον τεχνολογιών.