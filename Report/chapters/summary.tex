\pagestyle{plain}
\begin{center}
{\LARGE Περίληψη}\\[1cm]
\end{center}

\setlength{\parindent}{0pt}
Η παρούσα διπλωματική εργασία αποσκοπεί στην δημιουργία μίας εφαρμογής για τη συσκευή μικτής πραγματικότητας Microsoft HoloLens 2. Στόχος της είναι να υποστηρίξει άτομα με προβλήματα όρασης, συμβάλλοντας στην ανεξαρτησία τους και τη προσβασιμότητά τους. Ειδικότερα, προσφέρει τη δυνατότητα να περιηγηθούν με ασφάλεια σε έναν άγνωστο, για αυτούς, χώρο.

Αρχικά, περιγράφεται το θεωρητικό και τεχνολογικό υπόβαθρο. Πιο συγκεκριμένα, παραθέτουμε τα αίτια που οδηγούν σε απώλεια όρασης, τον αντίκτυπο που έχει αυτή στο άτομο και τις διαθέσιμες λύσεις για την αντιμετώπισή της. Επιπλέον, επεξηγούνται οι έννοιες της εικονικής, επαυξημένης και μικτής πραγματικότητας και αναλύονται τα τεχνικά χαρακτηριστικά της συσκευής Microsoft HoloLens 2 και οι δυνατότητές της, όπως η χωρική χαρτογράφηση και ο χωρικός ήχος, καθώς και τα εργαλεία που αξιοποιήθηκαν για την ανάπτυξη της εφαρμογής, όπως η Unity.

Στη συνέχεια, παρουσιάζεται η υλοποίηση της εφαρμογής, ξεκινώντας με το σχεδιασμό αυτής και τους περιορισμούς που τέθηκαν και, έπειτα, με τον τρόπο οργάνωσης και προετοιμασίας του project. Ακολουθεί μια αναλυτική περιγραφή όλων των λειτουργιών της εφαρμογής, από τις οποίες κύριες είναι ο εντοπισμός εμποδίων και η προειδοποίηση χρηστών, και ο τρόπος με τον οποίο αναπτύχθηκαν.

Η αξιολόγηση της εφαρμογής πραγματοποιήθηκε με τη βοήθεια δέκα εθελοντών, οι οποίοι κλήθηκαν να τη δοκιμάσουν παράλληλα με ένα αρκετά διαδεδομένο εργαλείο για άτομα με προβλήματα όρασης, το μπαστούνι. Από τις δοκιμές χρήσης και των δύο εργαλείων από τους εθελοντές, καταφέρνουμε να συλλέξουμε σημαντικά συγκριτικά δεδομένα, τα οποία και αναλύονται μαζί με αυτά που αντλούμε από τις απαντήσεις των ερωτηματολογίων και τις παρατηρήσεις των χρηστών. Με βάση τα δεδομένα αυτά, συμπεραίνουμε ότι, παρά τη χρηστικότητα της εφαρμογής, αυτή δεν καταφέρνει να βοηθήσει τους εθελοντές να ολοκληρώσουν το task τους ταχύτερα και με μεγαλύτερη ασφάλεια σε σχέση με το μπαστούνι. Για το λόγο αυτό, θα αναλυθούν και τα ελαττώματα της εφαρμογής, όπως αυτά παρατηρήθηκαν από εμάς και αναφέρθηκαν από τους εθελοντές και πως αυτά οδήγησαν στο προαναφερθέν αποτελέσμα.

Τέλος, η εργασία ολοκληρώνεται παραθέτοντας πιθανές μελλοντικές προεκτάσεις και βελτιώσεις της εφαρμογής με στόχο μια ακόμη καλύτερη εμπειρία για τον τελικό χρήστη.
\\[\baselineskip]
\textbf{Λέξεις-κλειδιά}: {\keywords}