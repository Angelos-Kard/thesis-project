Η υλοποίηση της εφαρμογής, τμηματοποιημένη

\subsection{Εισαγωγή}\label{subsec:developIntro}
Περιγραφή του development phase
Κώδικας των components στο παράρτημα Α μαζί με σχόλια
Όλο το project στο GitHub
Προγραμματισμός στο σπίτι και στο εργαστήριο

\subsection{Προετοιμασία και Εγκατάσταση}\label{subsec:developSetup}
Εργαλεία που χρησιμοποιήθηκαν και οι εκδόσεις τους (Unity, MRTK, Visual Studio, Hololens emulator)

\subsection{Οργάνωση GameObjects και Components}\label{subsec:developOrganization}
Ποια είναι τα GameObjects που χρησιμοποιήσαμε και ποια τα componetns που δημιουργήσαμε
Πως επιλέξαμε ποια components θα πάνε σε ποια game objects

\subsection{Εντοπισμός Εμποδίων}\label{subsec:developObstacleDetection}
Χωρική χαρτογράφηση (χαρτογράφηση γραφείου)
Raycasts και Boxcasts

\subsection{Προειδοποίηση Χρήστη}\label{subsec:developWarning}
Χωρικός ήχος
Χρήση πολλών ειδοποιήσεων - περιορισμός σε ένα
Τοποθέτηση του ήχου στο ύψος του αυτιού
Χρήση φωτός
Χρήση χειριστηρίου

\subsection{Φωνητικές Εντολές}\label{subsec:developVoiceCommands}
Χρήση φωνητικών εντολών για διάφορες λειτουργίες