Κατά τον σχεδιασμό της εφαρμογής, καθώς και κατά τη διάρκεια υλοποίησης αυτής, υπήρξαν πληθώρα περιπτώσεων, όπου κληθήκαμε να λάβουμε ιδιαίτερα σημαντικές αποφάσεις, που επηρέασαν σημαντικά την πορεία ανάπτυξης, λόγων των συνθηκών και προδιαφραφών που τέθηκαν ή των περιορισμών και δυσκολιών που αντιμετωπίσαμε.

Αρχικά, όσον αφορά την προετοιμασία και την εγκατάσταση των εργαλείων, ήρθαμε αντιμέτωποι με ένα ιδιαίτερα λειψό εγχειρίδιο (documentation)~\cite{hferrone_mixed}, το οποίο παρέχει η Microsoft.
Ειδικότερα, το documentation είναι πλούσιο σε πληροφοριές, που βοηθούν προγραμματιστές, οι οποίοι έρχονται πρώτη φορά σε επαφή με το headset, να κατανοήσουν πλήρως έννοιες και τεχνολογίες που αξιοποοιούνται από τη συσκευή, καθώς και τον τρόπο λειτουργίας της. Ωστόσο, σε προγραμματιστικό επίπεδο, οι πληροφορίες είναι ελάχιστες, ενώ οι επεξηγήσεις αντικειμένων, κλάσεων και συναρτήσεων είναι ιδιαίτερα ελλιπείς, καθώς περιορίζονται σε επιδερμικές περιγραφές αυτών και την παρουσιασή παραδειγμάτων και σε ένα πολύ μικρό δείγμα αυτών.
Παράλληλα, η Microsoft παρέχει μαθήματα (courses) για την πρακτική εκμάθηση των προγραμματιστών στην ανάπτυξη εφαρμογών για τη συσκεύη με τη χρήση του περιβάλλοντος Unity. Όμως, και σε αυτή την περίπτωση, τα courses περιορίζονται στη χρήση έτοιμων projects, όπου η μόνη συμβολή του χρήστη στη ανάπτυξη αυτών είναι η αλλαγή ορισμένων απλών ρυθμίσεων. Επομένως, κατά τη διάρκεια της ανάπτυξης (development), αναγκαστήκαμε να καταφύγουμε σε οδηγούς (tutorials) και forums, τα οποία αναπτύχθηκαν από τρίτους, από κοινότητες προγραμματιστών που διέθεταν εμπειρία πάνω στη συγκεκριμένη τεχνολογία.
Τέλος, ιδιαίτερο πρόβλημα αποτέλεσε η ασυμβατότητα (incompatibility) μεταξύ των διαφορετικών εκδόσεων των εργαλείων που χρησιμοποιήθηκαν (Unity, MRTK, Visual Studio), το οποίο καθυστέρησε την έναρξη της φάσης ανάπτυξης (development phase).

Επιπλέον, λόγω της ιδιαιτερότητας των χρηστών προς τους οποίους απευθύνεται η εφαρμογή, εκ πρώτης όψεως, φαίνεται αρκέτα οξύμωρη η επιλογή μίας συσκευής που ενσωματώνει την τεχνολογία Μικτής Πραγματικότητας, όπου, επί τον πλείστον, οι εφαρμογές απαιτούν από τους χρήστες να αλληλεπιδράσουν με εικονικά αντικείμενα και η όραση αποδεικνύεται εξαιρετικά σημαντική. Παρ' όλα αυτά, η συσκευή επιλέχθηκε λόγω compact σχεδιασμού, ενσωματώνοντας πληθώρα αισθητήρων σε μια μικρή, φορητή συσκευή. Επίσης, ο προγραμματιστής έχει τη δυνατότητα να ενσωματώσει φωνητικές εντολές, όπου οι χρήστες μπορούν χρησιμοποιήσουν για να αλληλεπιδράσουν με τις λειτουργίες της εφαρμογής.

Πέρα από τις δυνατότητες που προσφέρει η συσκευή, καθιστώντας την ιδανική επιλογή, θέτει, παράλληλα, έχει και ορισμένους περιορισμούς. Συγκεκριμένα, οι περιορισμοί αφορούν τις συνθήκες του περιβάλλοντος, όπου χρησιμοποιείται η συσκευή~\cite{dorreneb_2022_hololens}. 
Για την αποδοτική λειτουργία της συσκευής, πρέπει ο χώρος να είναι καλά φωτιζόμενος, να μην είναι ιδιαίτερα απλός, αλλά, αντιθέτως, να διαθέτει πολλά, μοναδικά αντικειμένα, τα οποία θα εξυπηρετήσουν στην καλύτερη αναγνώριση των αντικειμένων και των εμποδίων κατά τη χωρική χαρτογράφηση. Τέλος, οι συχνές αλλαγές και μετακινήσεις στο χώρο, καθώς και η ύπαρξη ανακλαστικών επιφανειών μπορούν να επηρεάσουν αρνητικά το tracking και τη χαρτογράφηση της συσκευής.
Για τους ανωτέρω λόγους, αποφασίστηκε η χρήση της εφαρμογής να περιοριστεί σε εσωτερικούς χώρους.

% Αδυναμία ανθρώπου να αντιληφθεί προέλευση ήχου στον κάθετο άξονα