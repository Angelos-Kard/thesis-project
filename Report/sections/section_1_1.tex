Η παρούσα διπλωματική εργασία χωρίζεται σε 5 κεφάλαια. Στο 2ο κεφάλαιο παρουσιάζεται σε βάθος το θεωρητικό και τεχνολογικό υπόβαθρο, το οποίο σχετίζεται με την ανάπτυξη της εφαρμογής. Έπεοτα, στο 3ο κεφάλαιο, περιγράφεται αναλυτικά η διαδικάσια σχεδιασμού και υλοποίησης της εφαρμογής, ενώ, στο 4ο κεφάλαιο, η εφαρμογή διατίθετε σε χρήστες, με σκοπό να την αξιολογήσουν. Παρατίθονται οι συνθήκες κάτω από τις οποίες έγινε η χρήση της εφαρμογής, ο τρόπος αξιολόγησης αυτής, καθώς και τα αποτελέσματα που προέκυψαν από τα πειράματα. Τέλος, στον 5ο κεφάλαιο, αναφέρονται ορισμένες από τις δυνατότητες και προεκτάσεις, που θα μπορούσε να αποκτήσει η εφαρμογή με την περαιτέρω ανάπτυξή της και την ενσωμάτωση επιπλέον τεχνολογιών.