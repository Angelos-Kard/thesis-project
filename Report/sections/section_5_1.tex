Ο κύριος στόχος της διπλωματικής εργασίας ήταν η υλοποίησης μιας εφαρμογής, η οποία θα προσέφερε ουσιαστική βοήθεια σε άτομα με προβλήματα	στην περιήγηση τους σε έναν άγνωστο χώρο. Η ανάπτυξη της βασίζεται στην αξιοποίηση σύγχρονων τεχνολογιών, όπως είναι η Μικτή Πραγματικότητα, η οποία, μέχρι και σήμερα, δεν έχει καταφέρει να εισέλθει σε μεγάλο βαθμό στην καθημερινότητα του μέσου ανθρώπου. Η χρήση της συσκευής Microsoft HoloLens 2 αποτέλεσε ιδανική επιλογή, διότι διαθέτει πληθώρα αισθητήρων απαραίτητων για τους σκοπούς της εφαρμογής μας σε μια φορητή συσκευή, μικρού σχετικά σχετικά μεγέθους. Έτσι μπορέσαμε να επικεντρωθούμε στην ανάπτυξη λογισμικού χωρίς να υπάρχει η ανάγκη δημιουργίας εξειδικευμένου hardware. Επιπλέον, αποτελεί ευκαιρία ώστε άτομα με προβλήματα όρασης να βιώσουν - εν μέρει - την εμπειρία της Μικτής Πραγματικότητας, η οποία γίνεται περισσότερο προσβάσιμη προς αυτούς. %chktex-file 8

Η εφαρμογή, η οποία δημιουργήθηκε, διαθέτει τις απαραίτητες λειτουργίες, ώστε να αποτελέσει ένα πρωτότυπο και να παρουσιαστεί η χρηστικότητά της, ωστόσο βρίσκεται σε ένα αρχικό στάδιο ανάπτυξης. Επιδέχεται πλήθος αναβαθμίσεων, οι οποίες θα μπορούσαν να βελτιώσουν την απόδοσή της, καθώς και την εμπειρία του χρήστη. Μεταξύ των άλλων, μεγαλύτερο ενδιαφέρον παρουσιάζει η χρήση τεχνητής νοημοσύνης σε συνδυασμό με υπολογιστική όραση με σκοπό, όχι μόνο τον καλύτερο εντοπισμό εμποδίων, αλλά και για την πρόβλεψη πιθανών συγκερούσεων όπως και για την αναγνώριση αντικειμένων και εμποδίων. Προσφέροντας στο χρήστη τη γνώση για το εμπόδιο που βρίσκεται στο δρόμο του, δηλαδή αν πρόκειται για ένα τοίχο, μια καρέκλα, έναν άνρθωπο ή μια πόρτα, του δίνεται η δυνατότητα να προσαρμόσει ανάλογα τον τρόπο αντιμετώπισης αυτής της δυσκολίας, καθώς και να κινείται με μεγαλύτερη ελευθερία στο χώρο, γνωρίζοντας τι υπάρχει σε αυτόν ή από που να μεταβεί σε κάποιο άλλο χώρο. Επίσης, χρήσιμη θα ήταν η δυνατότητα σύνδεσης του κινητού τηλεφώνου του χρήστη με την εφαρμογή μέσω ενός companion app, καθιστώντας τη διαχείριση αυτής και των λειτουργιών της ευκολότερη λόγω των επιλογών προσβασιμότητας που προσφέρει, όπως είναι οι screen readers. 
Με αυτό τον τρόπο, ο χρήστης δε θα χρειάζεται να βασίζεται αποκλειστικά στη χρήση φωνητικών εντολών.
% Παράλληλα, θα μπορούσε να λειτουργεί σα συσκευή μετάδοσης του haptic feedback